\documentclass[conference]{IEEEtran}
% Add the compsoc option for Computer Society conferences.
%
% If IEEEtran.cls has not been installed into the LaTeX system files,
% manually specify the path to it like:
% \documentclass[conference]{../sty/IEEEtran}

% Some very useful LaTeX packages include:
% (uncomment the ones you want to load)


% *** CITATION PACKAGES ***
%
%\usepackage{cite}
% cite.sty was written by Donald Arseneau
% V1.6 and later of IEEEtran pre-defines the format of the cite.sty package
% \cite{} output to follow that of IEEE. Loading the cite package will
% result in citation numbers being automatically sorted and properly
% "compressed/ranged". e.g., [1], [9], [2], [7], [5], [6] without using
% cite.sty will become [1], [2], [5]--[7], [9] using cite.sty. cite.sty's
% \cite will automatically add leading space, if needed. Use cite.sty's
% noadjust option (cite.sty V3.8 and later) if you want to turn this off.
% cite.sty is already installed on most LaTeX systems. Be sure and use
% version 4.0 (2003-05-27) and later if using hyperref.sty. cite.sty does
% not currently provide for hyperlinked citations.
% The latest version can be obtained at:
% http://www.ctan.org/tex-archive/macros/latex/contrib/cite/
% The documentation is contained in the cite.sty file itself.


% *** GRAPHICS RELATED PACKAGES ***
%
\usepackage{graphicx}



% *** MATH PACKAGES ***
%
%\usepackage[cmex10]{amsmath}
% A popular package from the American Mathematical Society that provides
% many useful and powerful commands for dealing with mathematics. If using
% it, be sure to load this package with the cmex10 option to ensure that
% only type 1 fonts will utilized at all point sizes. Without this option,
% it is possible that some math symbols, particularly those within
% footnotes, will be rendered in bitmap form which will result in a
% document that can not be IEEE Xplore compliant!
%
% Also, note that the amsmath package sets \interdisplaylinepenalty to 10000
% thus preventing page breaks from occurring within multiline equations. Use:
%\interdisplaylinepenalty=2500
% after loading amsmath to restore such page breaks as IEEEtran.cls normally
% does. amsmath.sty is already installed on most LaTeX systems. The latest
% version and documentation can be obtained at:
% http://www.ctan.org/tex-archive/macros/latex/required/amslatex/math/





% *** SPECIALIZED LIST PACKAGES ***
%
%\usepackage{algorithmic}
% algorithmic.sty was written by Peter Williams and Rogerio Brito.
% This package provides an algorithmic environment fo describing algorithms.
% You can use the algorithmic environment in-text or within a figure
% environment to provide for a floating algorithm. Do NOT use the algorithm
% floating environment provided by algorithm.sty (by the same authors) or
% algorithm2e.sty (by Christophe Fiorio) as IEEE does not use dedicated
% algorithm float types and packages that provide these will not provide
% correct IEEE style captions. The latest version and documentation of
% algorithmic.sty can be obtained at:
% http://www.ctan.org/tex-archive/macros/latex/contrib/algorithms/
% There is also a support site at:
% http://algorithms.berlios.de/index.html
% Also of interest may be the (relatively newer and more customizable)
% algorithmicx.sty package by Szasz Janos:
% http://www.ctan.org/tex-archive/macros/latex/contrib/algorithmicx/



% correct bad hyphenation here
\hyphenation{op-tical net-works semi-conduc-tor}


\begin{document}
%

\title{Behavioral Correlation: A new approach for clustering sensors in Wireless Sensor Networks}

% author names and affiliations
% use a multiple column layout for up to three different
% affiliations
\author{\IEEEauthorblockN{Fernando Rodrigues}
\IEEEauthorblockA{University of Fortaleza\\Fortaleza - Ceara - Brazil\\
fernandorodrigues@edu.unifor.br}
\and
\IEEEauthorblockN{Angelo Brayner}
\IEEEauthorblockA{University of Fortaleza\\
Fortaleza - Ceara - Brazil\\
brayner@unifor.br}
\and
\IEEEauthorblockN{Jose E. Bessa Maia}
\IEEEauthorblockA{State University of Ceara\\
Fortaleza - Ceara - Brazil\\
jose.maia@uece.br}}


% make the title area
\maketitle


\begin{abstract}

Sensor clustering is an efficient strategy to reduce the number of messages
transmitted to the sink in a multi-hop Wireless Sensor Network. Thus far,
grouping sensors in clusters is implemented by applying the principle of spatial
locality, which may ensure spatial correlation for data sensed by sensors within
a given area. Nonetheless, in several scenarios, sensors that are not spatially
close to each other may have similar data reading patterns.
In this work we present a new approach to cluster sensors in WSN, denoted {\it
BCWSN} (Behavioral Correlation in WSN), based on the behave of recent historical
data collected by sensors. Instead of using the spatial distance among sensors
for clustering them, the proposed approach uses the concept of behavioral
correlation to group sensors, that takes into account the concepts of difference
in magnitude and trend of the time series formed by the sensed data.
Furthermore, two scheduling intra-clustering methods are tested: Representative
Nodes and Cluster Heads. These techniques, which incorporate the mechanism of
behavioral correlation, are compared with each other and also with the naive
strategy (without clustering or temporal correlation) and with an approach based
on temporal correlation.
In order to validate our approach, simulations with a prototype have been
conducted over real temperature data. Our simulation shows that with 5\% error
threshold, BCWSN can save the communication overhead by as much as 97.72\% over
naive strategy and 69.02\% over a temporal correlation approach, besides a
reduction of 45.12\% in total data sensing over the other two approaches
compared, while the RMSE remains roughly stable. Hence, we believe that our
proposal greatly increases the energy efficiency for WSNs, with little loss of
accuracy.

% The results point to gains of up to 97.72\% over naive strategy in terms of
% reduction in message transmission and 45,12\% of reduction in data sensing,
% while the RMSE remains roughly stable. To the best of our knowledge, we believe
% that our proposal brings gains in energy efficiency for WSNs.

% The best way to improve prediction accuracy is by decreasing prediction errors,
% using the same energy amount than the second version, but there is a trade-off
% between prediction accuracy and energy consumption.

\end{abstract}

% IEEEtran.cls defaults to using nonbold math in the Abstract.
% This preserves the distinction between vectors and scalars. However,
% if the conference you are submitting to favors bold math in the abstract,
% then you can use LaTeX's standard command \boldmath at the very start
% of the abstract to achieve this. Many IEEE journals/conferences frown on
% math in the abstract anyway.

% no keywords




% For peer review papers, you can put extra information on the cover
% page as needed:
% \ifCLASSOPTIONpeerreview
% \begin{center} \bfseries EDICS Category: 3-BBND \end{center}
% \fi
%
% For peerreview papers, this IEEEtran command inserts a page break and
% creates the second title. It will be ignored for other modes.
\IEEEpeerreviewmaketitle



\section{Introduction}
% no \IEEEPARstart


Sensors are devices used to collect data from the environment related to the
detection or measurement of physical phenomena. Sensors are limited in power,
computational capacity, and memory. Advances in wireless communication have
enabled the development of massive-scale wireless sensor networks (WSN). In a
WSN, sensors are usually scattered in the network and use low-power
communication channels. Thus, sensors disseminate collected data to a base
station, from where the information (query) was originally requested. Wireless
sensor networks (WSNs) have been widely used for environmental monitoring (e.g.,
traffic, habitat), industrial sensing and diagnostics (e.g., factory, supply
chains), infrastructure protection (e.g., water distribution), battlefield
awareness (e.g., multi-target tracking) and context-aware computing (e.g.,
intelligent home) applications.

In spite of advances in WSN technology, a critical key point is still the
energy consumption of sensor nodes. It is well known that communication among
sensors is the activity responsible for the bulk of the power consumption. By
reducing communication costs, energy may be drastically saved, consequently
increasing the WSN's lifetime. An effective strategy to reduce energy
consumption is thus to reduce the number of messages (sensed data) sent across
the network. Nevertheless, the less the number of sensed data is transmitted,
the lower the accuracy of results provided by a WSN is. Thus higher accuracy in
WSNs comes at a higher energy cost.

By now, it is well-known that data collected by WSN are strongly temporally
and/or spatial correlated \cite{Yoon2005, Chu2006}. The traditional spatial data
correlation is related to the idea that the physical proximity among sensors
leads to similar measurements (values) of sensed data, phenomenon known as
"principle of spatial locality". Thus, one can infer that from the capture of
some sensors readings (located in some regions of sensing space), it is possible
to obtain, approximately, the values of the readings of other sensors in its
surroundings. On the other hand, the temporal correlation indicates the various
readings of a sensor within a time interval have a certain approximation of
their values (principle of temporal locality). such a feature makes possible to
predict (with a certain margin of error) sensed values in the future based on
data collected in the past.

Grouping sensors in clusters is the main technique used to take advantage of the
principle of spatial locality for reducing the energy consumption in WSNs. This
is because one can use only a few representative nodes from each cluster to
sense data in a given spatial region (cluster) in which sensors are spatially
correlated.
Several works have been proposed in order to use that technique, with different
approaches \cite{Chu2006, Villas2012, Singh2010, Liu2007, Shah2007}.

Nonetheless, in several scenarios sensors, which are not spatially close to each
other, may have similar data reading patterns. In order to illustrate such a
claim consider a dense WSN deployed to monitor forest fires. 
Now, suppose a scenario in which the monitored region is affected by dozens of
small forest fires. Figure \ref{fig:contour_lines} depicts a possible temperature
contour lines graph for this hypothetical situation. Observe that the contour
lines in Figure \ref{fig:contour_lines} form several closed regions representing
areas which may have small forest fire areas, where it is very likely that the
temperature measurements of sensors in those spatially separated regions present
high correlation. For that reason, we claim that in such cases, a better
alternative would be to use sensor clustering strategy based on
\textit{Behavioral Correlation}.

The idea behind the concept of {\it Behavioral Correlation} is to identify
similar patterns of sensor readings even in sensors which are geographically
distant from each other. Thus, one could apply a Behavioral Correlation
Clustering (BCC) technique to group sensors which are spatially separated into a
single cluster, in contrast to existing spatio-temporal correlation techniques.
The BCC technique clusters sensors for which the forecasting models of sensed
data time series are approximately the same independent of spatial proximity of
sensors.

\begin{figure}[!htb]
\centering
	\includegraphics[scale=0.7]{I2.png}
    \caption{Contours lines of temperature}
    \label{fig:contour_lines}
\end{figure}

In this sense, in this paper, we present new approach for clustering sensors in
WSNs. The main features of the proposed approach are: {\it (i)} Cluster
formation based on the \textit{Behavioral Correlation} of the sensors, which, in
turn, is computed from the time series of sensor readings by applying a
\textit{Similarity Measure}, and; {\it (ii)} the use of a linear regression
model for the temporal suppression of sensed data through the maximum error
level (threshold) desired by the user used to control the data to be sent to the
sink. Hence, special sensor nodes only transmit data which are novelties for the
regression model applied by our proposal. Furthermore, two different approaches
to select of active nodes (scheduling) in each cluster have been implemented:
Representative Nodes (RNs) and Cluster Heads (CHs).


The remainder of this document is organized as follows. Section
\ref{related-work} review related work and point out the differences between our
method and other existing methods. We propose the Behavioral Correlation in WSN
(BCWSN) method and explain the two sensor scheduling scheme based on it in
Section \ref{implementing-bcwsn}. In Section \ref{data-and-experiments}, we
describe the condictions of performance evaluation of four different approaches
compared. The results of simulations with a SinalGo \cite{Sinalgo2007} prototype
operating over real sensor data are reported in Section
\ref{results-and-discussion}. Finally, Section \ref{conclusion} concludes the
paper.


\section{Related Work}
\label{related-work}

In \cite{Vuran2004}, proposes a strategy to cluster sensors in WSNs. The idea is
the following: given a set of N sensors, M nodes, with $M < N$, are chosen to
send data. The M representative nodes are defined based on the application of a
distortion function ($D(M)$) on sensed data.
The spatial distance between the nodes (representative) directly influences the
computation of the distortion function by means of a correlation coefficient.
That work does not take into account the energy capacity of each node as a
criterion for choosing representative nodes, although this is a very important
factor due to the restrictive characteristics regarding the energy consumption
of the nodes in a WSN.

In EAST \cite{Villas2012}, sensors are grouped into two levels, under a spatial
correlation approach, while the leader and the representative nodes perform a
temporal suppression technique.
The leader node generates a representative value for each cluster based on data
received by the representative nodes, which form a subset of all the nodes that
sense the same event. The sensed area is divided into "event areas", which in
turn are divided into "correlation regions (c) or cells", where the formers will
be managed, each one, for a "Coordinator node" and the "correlation areas" will
be represented, one by one, by a "Representative node" because a single reading
within this region is enough to represent it.
The size of the correlation region (c) can be decremented or incremented by the
sink according to the application and the characteristics of the event, to
maintain the accuracy of the data collected.

Another way to group sensor nodes into clusters is through measures of
dissimilarity.
In EEDC \cite{Liu2007}, such measures of dissimilarity are calculated by the
sink node for all pairs of nodes of the network, regardless of their location.
The measure of dissimilarity between two nodes is calculated based on up to $3$
parameters, namely:
the differences in magnitude (\textit{M}) and trend (\textit{T}) of the data
values and the geographical/euclidean distance between nodes ($g_{max}dist$).
The criterion of formation of clusters is based on the maximum threshold of
dissimilarity (max\_dst) defined by a tuple (\textit{M}, \textit{T},
$g_{max}dist$), based on the measure of dissimilarity between the nodes. It
works as follows: $1)$ Initially, the data sensed by each node are sent in the
form of a temporal series for the sink. $2)$ The sink then stores all the data
from the sensors and then calculates the measure of dissimilarity (previously
mentioned) for each pair of nodes of the network. $3)$ With the measures
calculated and the maximum threshold of dissimilarity (max\_dst), the sink
divides the nodes into clusters. 
That work does not takes into account the energy reserve of each node as a
criterion for choosing representative nodes (it is used an algorithm that makes,
simultaneously, the equitable scheduling - round robin - along with the random
choice of representatives nodes).

The spatial correlation through the formation of clusters is addressed in
\cite{Pham2010} in the form a flooding algorithm where the sink node starts to
send messages to the other nodes of the network, inviting them to form groups
from criteria such as a dissimilarity measure, in addition to the physical
proximity between nodes, since, of course, the message forwarding in a WSN
occurs between adjacent nodes (i.e. geographically close). Cluster Heads (CHs)
are selected, basically, by 2 parameters: {\it (i)} the nodes that are one hop
from the ancestor that sent the message calculate the measure of dissimilarity
with the mean value informed in the message and then those that are within the
threshold of dissimilarity if they apply for CH, where {\it (ii)} it is said to
be the CH the one that have higher level of energy reserve.
We should notice that, during the process of forming clusters, the nodes that
will form the communication backbone between each Cluster Head node and the sink
are also configured. A scheduling of each cluster is done through round robin in
order to decide which member node that will be active in each time slot making
the sensing and sending the data to its respective CH.
The weak point is the process of forming clusters, in
which there is an intensive exchange of messages, scattering a significant
amount of energy from the network sensors.

In \cite{Shah2007}, the spatial correlation is explored by a mechanism called
the GSC (Gridiron Spatial Correlation), where the sensed region has a Cluster
Head that will be in the center of the region delimited by r (radius of the
monitored region), which will be divided into correlated regular regions
(quadratic), according to the spatial density level chosen, defined through
$\theta$ (size of the correlation region equal to $\theta^2$). In this way,
active sensors will be chosen according to 2 basic parameters: {\it (i)} the
proximity of the them regarding the center of the regions correlated and {\it
(ii)} their energy level must be within a certain threshold, above the ones of
their closest neighbours. The scheduling of active nodes works through the
passage of a list by the cluster-head for all nodes with the nodes being active
in each time slot, where this configuration is only changed when one of the
active nodes has its energy level below the threshold established.
That work does not describe how the energy threshold is calculated neither how
this reconfiguration of the sizes of the rectangles are done and not even gives
examples of the that.


\section{Implementing Behavioral Correlation in WSNs}
\label{implementing-bcwsn}

In this section, we present the proposed approach for clustering sensors in a
WSN based on the notion of {\it Behavioral Correlation}. Our approach, called
Behavioral Correlation in Wireless Sensor Networks (\textit{BCWSN} for short),
is based on clustering sensors by means of behavioral correlation (descrived in
\ref{clustering-sensors}), which is in turn computed from the time series of
sensor readings, and on using a linear regression model for the temporal
suppression of data to be sent to the sink node.

Two different approaches for intra-cluster sensor node scheduling have been
implemented in order to uniformly distribute the sensing activity for the
sensor nodes of a cluster. In the first one, called Representative Nodes (RNs),
only one node in each cluster $C_{i}$ is chosen at each time to represent that
cluster. In other words, a representative node is responsible for sensing and
predicting data in $C_{i}$ for given time interval. In this case, we have a
greater energy economy in the network as a trade-off from a little
general precision, because changes in fenomena sensed at network positions where
the sensors are in stand-by mode (non-Representative Nodes) would not be
captured. To stand up to this weakness, we have a second method, called
Cluster Heads (CHs). In this other approach, on the other hand, one sensor node
(the CH) in each cluster $C_{i}$ is selected to coordinate the data sensing
activity carried out by all nodes in $C_{i}$. So, changes are promptly sensed by
active nodes, having major impact on energy expediture from nodes activation.

\subsection{The BCWSN Mechanism}


The algorithm BCWSN can be divided into six steps described next.


\subsubsection{Learning Stage}

In this step, the sink node collects sensed data from all sensors belonging to
the network in order to compute the initial cluster formation and the
coefficients of the linear regression equation (see Section \ref
{data-predict}). Thus, the sink node firstly sends a broadcast message to all
nodes of the network, requesting the following data from sensors:
battery level, spatial location and sensed values. The amount of data for the
learning stage is a parameter, denoted initial slot time, which should be
defined by the application expert.


\subsubsection{Clustering Sensor Nodes}
\label{clustering-sensors}

The calculation of \textit{Similarity Measures} between the readings of sensors,
based on which described in \cite{Liu2007}, is accomplished by the sink node, as
criterion for clustering, from 2 (or 3) parameters (the third is
optional):
the similarity of magnitude, the similarity of trend and the distance
(\textit{maxDistance}) between nodes (in our case, not mandatory). So, we define
the metrics below to evaluate if two sensor are in the same cluster or not:

\newtheorem{defini}{Definition}

\begin{defini}
Similarity of magnitude-M: Two sensors ($S_{1}$ and $S_{2}$) with time series
$S_{1}$\{$s_{1_1}$,$s_{1_2}$,\ldots,$s_{1_n}$\} and
$S_{2}$\{$s_{2_1}$,$s_{2_2}$,\ldots,$s_{2_m}$\} are magnitude-M similar if for
{\it i} ($1\leq i\leq n$), {\it j} ($1\leq j\leq m$) such that ($i = j$) for x times,
then 
\begin{equation}
\label{equ:magni}
\frac{\sum_{i = 0}^{x} |s_{1_i}-s_{2_i}|}{x} \leq M
\end{equation}
%$\displaystyle \sum_{i = 0}^{n} |s_{1_i}-s_{2_i}| \leq M$
\end{defini}

\begin{defini}
Similarity of trend-T: Two sensors ($S_{1}$ and $S_{2}$) with time series
$S_{1}$\{$s_{1_1}$,$s_{1_2}$,\ldots,$s_{1_n}$\} and
$S_{2}$\{$s_{2_1}$,$s_{2_2}$,\ldots,$s_{2_n}$\} are trend-T similar if 
\begin{equation}
\label{equ:trend}
\frac{n_{1}}{n} \geq T,
\end{equation}
where $n$ is the total number of sensed data and $n_{1}$ is the number of pairs
$(s_{1_i},s_{2_i})$ in the time series that satisfy $\nabla s_{1_i} \* \nabla
s_{2_i} \geq 0$, such that $\nabla s_{1_i} = s_{1_i} - s_{1_{i-1}}$, $\nabla
s_{2_i} = s_{2_i} - s_{2_{i-1}}$ and $1 < i \leq n$.
\end{defini}

Thus, from the time series of readings of each two sensors, sink performs the
comparison of data received through the calculations of similarity among that
sensors (peer to peer) as follows:
If the average of the differences between the values of the sensors readings is
less than or equal to M-magnitude similarity parameter (\ref{equ:magni}) and if
the percentage similar trends (growth/decay) of the data of the two sensors is
greater or equal to the T-trend similarity (\ref{equ:trend}), so these sensors
will be in the same cluster.
Otherwise, they will be in different clusters.

Thus, at the end of this process, all sensors nodes have been classified in
clusters by the sink, according to initial readings sent by sensor nodes to the
sink and the \textit{Similarity Measures} (based on M and T parameters, seted by
the expert), using the \textit{behavioral correlation} proposed.
Whenever the sink receives a notification of novelties threshold exceeded of a
sensor node (that can be a RN or a CH), it (sink) recalculates the similarity
measures between all nodes of this cluster to check if it is necessary to
perform a split in such a cluster.

Once the sensors are initially grouped into clusters, Representative Nodes or
Cluster Heads of each cluster will be chosen through 2 criteria, applied to all
nodes of each sensor clusters:
\begin{enumerate}
    \item The nodes are ordered by the criterion of highest energy level (higher
    battery load), largest to smallest
    \item The nodes that have equally the highest energy level, will then be
    classified in according to their distance to the sink, the shortest distance
    to the biggest
 \end{enumerate}


After this classification process, the first one of each cluster node (that is,
the one that has the higher energy level of battery and the shortest distance to
the sink) will be initially chose as Representative Node / Cluster Head. This
process of choosing of RNs / CHs is repeated for each cluster, whenever a new
notification is received by the sink.


\subsubsection{In-Network Data Prediction}
\label{data-predict}


The in-network prediction implemented by BCWSN relies on the
following linear regression equation: $\hat{S}(t) = a + bt$.
The time $t$ is an independent variable. $\hat{S}(t)$ represents the estimated
value of $S(t)$ and is variable with $t$. Parameter $a$ is the interceptor-t
(value of $\hat{S}(t)$ for $t=0$) and $b$ is the stretch slope, and are computed
as follows:
\begin{equation}
	a = \frac{1}{N}\left(\sum S_{i} - b\sum t_{i} \right) = \bar{S} - b\bar{t},
\end{equation}
\vspace*{-.3cm}
\begin{equation}
	b = \frac{\sum \left(t_{i} - \bar{t}\right)\left(S_{i} - \bar{S}\right)}{\sum \left(t_{i} - \bar{t}\right)^{2}}.
\end{equation}

The idea behind this method is that both the sink and the sensor node know the
regression equation to predict the sensed values, such that, if the value read
by the sensor is within the limit stipulated maximum error (error threshold),
then the sensor will not need to send data to sink (as it is able to calculate
the regression equation through a good approximation for the values), and thus
the network is saving power of the sensors because they do not need to send such
messages \cite{MaiaACR2013}.

After the calculation of the coefficients from regression equation for each
corresponding node (depending of the case, if Representative Nodes, only this
nodes will receive the coefficients; otherwise, if Cluster Heads, all sensors
from all clusters will receive the coefficients), the sink sends this data
through a message addressed to the respective node, who, in turn, should begin
the process of sensing (reading/prediction loop).

In case of RN, the sink must also estimate the maximum amount of predictions to
be performed by each Representative Node until it notifies the sink the end of
its time as a representative node. This is done to place the energy balance of
sensors in each cluster.

 
\subsubsection{Sensing}

As already mentioned, two different approaches for intra-cluster sensor node
scheduling have been implemented in order to uniformly distribute the sensing
activity for the sensor nodes of a cluster. In the first one, called
Representative Nodes (RNs), only one node in a cluster $C_{i}$ is responsible
for sensing and predicting data in $C_{i}$ for given time interval. In the
second method, called Cluster Heads (CHs), on the other hand, one sensor node
(the CH) in each cluster $C_{i}$ is selected to coordinate the data sensing
activity carried out by all nodes in $C_{i}$.

So, the nodes that have received coefficients enter into reading / prediction
loop, in such a way that every step, each one must verify that the read value
(sensing) is within the bound of error (threshold) to the value
predicted by calculating the regression equation from that node for that time,
acoording the linear regression equation in \ref{data-predict}.
If the limits of error are respected, then the node's regression equation
encountered a hit; otherwise, it must record an error (miss). If the number of
errors (misses) exceeds a certain limit (pre-set), then such node must report
this fact.

In RN mode, only the Representative Nodes (one for cluster) will receive the
coefficients and execute the process of sensing (reading/prediction loop).
In case of a number of errors (misses) greater than the limit error, the RN must
report the sink directly with "Novelties".
Otherwise, the Representative Node must perform such process of
reading/verifying the prediction values to the limit of predictions made for
that Representative Node by sink, calculated in the previous phase.

In CH mode, all sensors (in each cluster) will receive their respective
coefficients and will execute the process of sensing (reading/prediction loop).
But, in this case, whenever occurs a number of errors (misses) greater than the
limit error (per sensor) in a sensor node, that sensor send a
notification for its respective Cluster Head. This in turn record the number of
notifications received, in such way that when the number of notifications
received from any sensor from that cluster exceeds a pre-set limit (limit per
cluster), then the CH sends a message to the sink with the "Novelties".

\subsubsection{Processing of "Novelties"}

During the execution of the sensing loop, when a special node (RN or CH)
verifies that the number of errors (notifications) exceeded a given threshold
(pre-set), this node must inform such event to the sink by sending a message,
which will contain still the latest data readings/sensings performed beyond the
current level of sensor battery. 
When such a message arrives at the sink node, this must update
the data relating the respective node (including the battery level) and then
must check one of the following possibilities: {\it (i)} If there is a need to
make a (new) division / split of the cluster, with the choice of new
Representative Nodes / Cluster Heads, or {\it (ii)} If there is no need of
division, but only the choice of a new Representative Node / Cluster Head for
the current cluster, depending on the level of RN / CH current battery in
relation to other sensor nodes that cluster, or {\it (iii)} If there is no need
nor of division neither the choice of a new RN / CH, simply that the sink
perform the calculation of new Coefficients (updated) of the regression equation
of such a RN / CH, and (in case of RN) still retain the maximum number of
prediction loops that RN (i.e. continuing to counter predictions without
restarting).
Whenever a RN / CH send a message to the sink (by reason of the novelties report
or by have reached the maximum number of loops of prediction), this message must
contain the current battery level information (energy reserves) of that node, in
such a way that the sink can update this information and verify the need to
exchange the sensor node to be chosen as RN / CH of the respective cluster.

\subsubsection{Cluster Automatic Reorganization}

The sink has the ability to reorganize the clusters automatically according to
the need. There are two basic forms of the sink reorganizing the clusters. When
the sink receives "Novelties" from a special node (RN or CH), it checks if the
sensor nodes grouped in that cluster are no longer meeting the minimum
requirements of similarity required from the parameter configuration data
(M-magnitude and T-trend), by receiving new data readings of sensors from that
cluster. In this case, the sink perform the split of that cluster in two or more
new clusters, selecting new Representative Nodes or Cluster Heads (according
with the case) - one for each of the new clusters - in accordance with the
criteria set out in \ref{clustering-sensors}. In addition to that, the sink must
keep a control on the minimun occupancy of sensors per cluster, in such a way
that the sink is enable of firing the merging of clusters if it verifies this
need. Hence, whenever there are a large number of divisions (splits) and the
minimum rate occupation clusters (average of nodes per cluster) on the network
is disregarded, the sink will fire a global process of regrouping (merge) or
restructuring of the network, such that all nodes can have their sensors read
data reassessed, and thus can be reconfigured into new clusters that represent
better the current state of the network. Thus, one avoid that, in the long term,
and consequently, with a increasing number of divisions (splitting) of sensor
clusters, it can cause the clusters become smaller, increasing the energy
expenditure of the sensors network.


\section{Data and Experiments}
\label{data-and-experiments}

In our tests, we done a comparative analysis between the following approaches:
1) On approach Naive \cite{Madden2005}, all sensors are network data reads
(sensing) and send immediately these data to the sink; 2) On approach using
Temporal Correlation (Adaga-P*) \cite{MaiaACR2013} \cite{MaiaSAC2013}, all nodes
are sensors to read data from the environment, but only those whose readings
submit values outside the margin of error (user-defined threshold) in relation
to predicted values by the use of a regression equation (whose coefficients are
calculated by sink and sent for each sensor), send your data (Novelties) to the
sink; 3) On approach BCWSN-RN, after the formation of the clusters based on the
\textit{Behavioral Correlation} proposed, only the Representative Nodes (RNs) of
each cluster perform reading and data (sensing) only those who detect a quantity
of "Novelties" above the minimum acceptable amount, configured by the user, shall
send such information (Novelty) to the sink; 4) in this latter approach,
referred to as BCWSN-CH, after the clustering (also based on the
\textit{Behavioral Correlation}), all sensors, including the Cluster Heads (CHs)
choosen, make sensing / prediction, but sending data ("Novelties") to the sink is
divided into two phases: At first, only sensors send "Novelties" to their
respective Cluster Head after the occurrence of a minimum threshold of "Novelties"
(say N) configured by the user; In the second phase, each Cluster Head, in turn,
only relays "Novelties" for sink after receiving a minimum number of notifications
from your sensors (let's say M), also configured by the user.
The approaches 1) and 2) were implemented based on the references. Our proposal
was implemented through the items 3) and 4).
To make the performance evaluation of our work, we use the SinalGo version
v.$0.75.3$ simulator \cite{Sinalgo2007}. The data used for the simulation were
extracted from the real data of experiment Intel Lab Data \cite{Intel2004}.


\section{Results and Discussion}
\label{results-and-discussion}

Our tests focused mainly on 3 parameters to evaluate the performance of our
solution (approaches BCWSN-RN and BCWSN-CH, both described in section
\ref{data-and-experiments}) compared to the other two approaches (Naive and
Adaga-P*): 1) The Root Mean Square Error (RMSE) was used to verify the quality
of the prediction values calculated in the sink with the data sensed values
actually read by the sensors - the smaller RMSE, better was the approximation in
sink of data read by the sensors; 2) The Total Number of Messages sent by all
sensors network, which is an important factor in the energy consumption of the
network - the lower this amount of messages, the greater lifetime of the
network, and 3) The Number of Data Readings (Sensing) performed by all sensor
nodes, since this number is also related to the energy expenditure of sensor
nodes, and as the messages, the lower this number, the more energy durability of
the network.

As can be observed in the tables \ref{tab:rmse}, \ref{tab:num-msg} and
\ref{tab:sens-read}, at the end of 1000 cycles execution of the simulation
algorithm, the approach 3 (BCWSN-RN) presented a RMSE of $0.5$ (with average
0.5283 and standard deviation 0,085), while we had an average of transmitted
messages $96.5\%$ smaller as well as an average of sensor readings $45.12\%$
lower than the first approach (Naive). Compared with the approach 2 (Adaga-P*),
approach 3 had an average of messages exchanged $52.51\%$ lower and an average
of sensor readings $45.12\%$ lower than the second approach, showed an average
RMSE only $21.6\%$ higher that approach 2.
With respect to approach 4 (BCWSN-CH), it presented an average RMSE of $0.337$,
with an average of messages exchanged $97.72\%$ smaller and an average of sensor
readings $19.41\%$ less the first approach (Naive). When compared to approach 2
(Adaga-P*), the approach 4 presented a RMSE $22.53\%$ smaller, with an average
of messages exchanged $69.02\%$ smaller and a number of sensor readings
$19.41\%$ less.

\begin{table}[h!]
\caption{Evaluation results - Approach x RMSE per Round (1000 cycles)}
\label{tab:rmse}
\begin{center}
\begin{tabular}{|l||l|l|l|l|}
\hline
RMSE &AVG &STD &MAX &MIN \\
\hline\hline
Naive &0 &0 &0 &0 \\
\hline
Adaga-P* &0.4344 &0.0393 &0.485 &0.285 \\
\hline
BCWSN-RN &0.5283 &0.0851 &1.538 &0.492 \\
\hline
BCWSN-CH &0.3365 &0.0217 &0.366 &0.229 \\
\hline
\end{tabular}
\end{center}
\end{table}

\begin{table}[h!]
\caption{Evaluation results - Approach x Number of Messages per Round (1000 cycles)}
\label{tab:num-msg}
\begin{center}
\begin{tabular}{|l||l|l|l|l|}
\hline
Num Msg &AVG &STD &MAX &MIN \\
\hline\hline
Naive &224 &0 &224 &224 \\
\hline
Adaga-P* &10.93 &10.89 &142 &0 \\
\hline
BCWSN-RN &8.48 &10.10 &55 &0 \\
\hline
BCWSN-CH &5.14 &9.30 &55 &0 \\
\hline
\end{tabular}
\end{center}
\end{table}

\begin{table}[h!]
\caption{Evaluation results - Approach x Sensor Readings per Round (1000 cycles)}
\label{tab:sens-read}
\begin{center}
\begin{tabular}{|l||l|l|l|l|}
\hline
Sensor Reading &AVG &STD &MAX &MIN \\
\hline\hline
Naive &53 &0 &53 &53 \\
\hline
Adaga-P* &53 &0 &53 &53 \\
\hline
BCWSN-RN &26.27 &51.24 &279 &0 \\
\hline
BCWSN-CH &41.95 &36.14 &276 &0 \\
\hline
\end{tabular}
\end{center}
\end{table}



% \begin{table}[h!]
% \caption{Evaluation results - Comparison between four approaches (at the end of 1000 cycles)}
% \begin{center}
% \begin{tabular}{|l||l|l|l|}
% \hline
% Approach (N = 54 sensors) &RMSE &\#Messages &\#Readings \\
% \hline\hline
% Naive &0 &222242 &52812 \\
% \hline
% Adaga-P* &0,467 &9340 &51970 \\
% \hline
% BTCWSN-RN &0,502 &8146 &27626 \\
% \hline
% BTCWSN-CH &0,355 &5043 &42149 \\
% \hline
% \end{tabular}
% \end{center}
% \end{table}
% 
% \begin{table}[h!]
% \caption{Naive x BTCWSN-RN (Percentage difference at the end of 1000 cycles)}
% \begin{center}
% \begin{tabular}{|l||l|l|l|}
% \hline
% Approach (N = 54 sensors) &RMSE &\#Messages &\#Readings \\
% \hline\hline
% Naive (A) &0 &222242 &52812 \\
% \hline
% BTCWSN-RN (B) &0,502 &8146 &27626 \\
% \hline
% Diff (A-B) &N/A &$-96,33\%$ &$-47,69\%$ \\
% \hline
% \end{tabular}
% \end{center}
% \end{table}
% 
% \begin{table}[h!]
% \caption{Adaga-P* x BTCWSN-RN (Percentage difference at the end of 1000 cycles)}
% \begin{center}
% \begin{tabular}{|l||l|l|l|}
% \hline
% Approach (N = 54 sensors) &RMSE &\#Messages &\#Readings \\
% \hline\hline
% Adaga-P* (A) &0,467 &9340 &51970 \\
% \hline
% BTCWSN-RN (B) &0,502 &8146 &27626 \\
% \hline
% Diff (A-B) &$+7,49\%$ &$-12,78\%$ &$-46,84\%$ \\
% \hline
% \end{tabular}
% \end{center}
% \end{table}
% 
% \begin{table}[h!]
% \caption{Naive x BTCWSN-CH (Percentage difference at the end of 1000 cycles)}
% \begin{center}
% \begin{tabular}{|l||l|l|l|}
% \hline
% Approach (N = 54 sensors) &RMSE &\#Messages &\#Readings \\
% \hline\hline
% Naive (A) &0 &222242 &52812 \\
% \hline
% BTCWSN-CH (B) &0,355 &5043 &42149 \\
% \hline
% Diff (A-B) &N/A &$-97,73\%$ &$-20,19\%$ \\
% \hline
% \end{tabular}
% \end{center}
% \end{table}
% 
% \begin{table}[h!]
% \caption{Adaga-P* x BTCWSN-CH (Percentage difference at the end of 1000 cycles)}
% \begin{center}
% \begin{tabular}{|l||l|l|l|}
% \hline
% Approach (N = 54 sensors) &RMSE &\#Messages &\#Readings \\
% \hline\hline
% Adaga-P* (A) &0,467 &9340 &51970 \\
% \hline
% BTCWSN-CH (B) &0,355 &5043 &42149 \\
% \hline
% Diff (A-B) &$-23,98\%$ &$-46,01\%$ &$-18,90\%$ \\
% \hline
% \end{tabular}
% \end{center}
% \end{table}

In the Fig. \ref{fig:rmse}, one can observe a graph with the evolution of RMSE
in four tested approaches. The RMSE in the BCWSN-RN and Adaga-P* approaches tend
to get very close from 0.5 at about 900 cycles, while the BCWSN-CH stabilizes at
about 0.4 in the same point. It is important to note that in the Naive approach,
the RMSE is always 0 (zero), because all sensors send the overall sensed data to
the sink.

\begin{figure}[!htb]
\centering
	\includegraphics[scale=0.085]{graf_RMSE_.png}
    \caption{RMSE per Round}
    \label{fig:rmse}
\end{figure}

In the Fig. \ref{fig:num-msg}, one can observe a graph with the evolution of the
number of transmitted messages ({\it "NumMsg"} for short) in tested approaches.
NumMsg in the Naive approach grows linearly faster than in the other approaches.
The BCWSN-CH is which has the lowest number of messages, followed by BCWSN-RN,
and finally the Adaga-P* approach.

\begin{figure}[!htb]
\centering
	\includegraphics[scale=0.085]{graf_numMsg_.png}
    \caption{Number of Messages per Round}
    \label{fig:num-msg}
\end{figure}

% \begin{figure}[!htb]
% \centering
% 	\includegraphics[scale=0.09]{graf_NumMsg_without_naive_.png}
%     \caption{Number of Messages per Round without Naive}
%     \label{fig:num-msg-without-naive}
% \end{figure}

Finally, the graph in Fig. \ref{fig:sens-reading} shows the evolution of
the data sensed readings between the 4 tested approaches.
Naive and Adaga-P* approaches, as one can observe, have the same number of data
sensed readings (53) per round, because in these approaches all sensors
make sensing in all cycles. In the BCWSN-CH, sensors do not make sensing while
waiting to receive new coefficients after sending novelties to sink, and in the
BCWSN-RN, only the Representative Nodes (one per cluster) make sensing.

\begin{figure}[!htb]
\centering
	\includegraphics[scale=0.085]{graf_SREAD_.png}
    \caption{Number of Data Sensed Reading per Round}
    \label{fig:sens-reading}
\end{figure}


\section{Conclusion}
\label{conclusion}

As result of our experiments, we show that the use of the \textit{Behavioral
Correlation} approach associated with a temporal correlation technique using two
different methods for intra-cluster scheduling, which are Representative Nodes
(RNs) and Cluster Heads (CHs), brought a great benefit to the energy economy of
Wireless Sensor Networks (WSNs), since, while it remained a low RMS Error for
the data obtained by the sink using such techniques (behavio-temporal
suppression), two important factors in energy expenditure of a sensor network
decrease, which are the total number of messages sent on the network and the
total number of sensor readings.


% conference papers do not normally have an appendix



\bibliographystyle{IEEEtran}
\bibliography{IEEEabrv,ISCC2014rodrigues_brayner_maia}  
% ISCC2014rodrigues_brayner_maia.bib is the name of the Bibliography in this case
% You must have a proper ".bib" file
%  and remember to run:
% latex bibtex latex latex
% to resolve all references

\end{document}


